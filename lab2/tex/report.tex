\documentclass[a4paper,12pt]{report}

\input{header.tex}


\title{Отчёт \\ по лабораторной работе №2 \\ на тему: \\ "Моделирование мьютексов" \\ по курсу: Математические основы верификации ПО}
\author{Жаровой Наталии Александровны, ИУ7-41М\\}

\begin{document}

\maketitle

\section*{Постановка задачи}

На языке Promela описать взаимодействие двух процессов, разделяющих ресурсы. После обнаружения гонки необходимо 
смоделировать использование мьютекса для предотвращения гонки.

\section*{Моделирование состояния гонки}
Для того чтобы смоделировать состояние гонки, создадим глобальный объект счётчика <<x>> и два
конкурирующих процесса. При этом процесс <<counter>> будет увеличивать счётчик на 1, а процесс <<printer>>
будет выводить текущее значение счётчика, в том случае если его значение нечётно.

\subsection*{Листинг программы}
\begin{Verbatim}
#define N 10

int x = 0
bool end = 0

proctype counter() {
    do
    :: (x > N) -> break
    :: x = x + 1
    od;
    end = 1
}

proctype printer() {
    do
    :: (end) -> break
    ::  if
            :: (x % 2) -> printf("x=%d\n", x)
            :: skip
        fi
    od
}

init {
    run printer();
    run counter();
}
\end{Verbatim}

Если несколько раз позапускать данную модель, можно увидеть, что не всегда выводятся только нечётные значения счётчика:
\begin{Verbatim}
          x=2
          x=11
          x=11
3 processes created
\end{Verbatim}

Используя возможность языка Promela <<atomic>>, определим проверку на нечётность и непосредственно вывод значения как атомарную последовательность.
Таким образом, второй поток не сможет поменять значение глобальной переменной между проверкой и выводом.

\subsection*{Листинг исправленной программы}
\begin{Verbatim}
#define N 10

int x = 0
bool end = 0

proctype counter() {
    do
    :: (x > N) -> break
    :: x = x + 1
    od;
    end = 1
}

proctype printer() {
    do
    :: (end) -> break
    ::  if
            :: atomic { (x % 2) -> printf("x=%d\n", x) }
            :: skip
        fi
    od
}

init {
    run printer();
    run counter();
}
\end{Verbatim}

Теперь сколько бы раз мы не запускали модель, всегда выводятся только нечётные значения счётчика.

\section*{Выводы}

В процессе выполнения работы была написана модель программы, допускающей состояние гонки. С использованием возможности языка Promela <<atomic>> 
гонка была устранена, путём объединения нескольких операций в неделимую последовательность.

\end{document}
