\documentclass[a4paper,12pt]{report}

\usepackage[utf8x]{inputenc}
\usepackage[T2A]{fontenc}
\usepackage[english, russian]{babel}
\usepackage{doxygen}


% Опционно, требует  apt-get install scalable-cyrfonts.*
% и удаления одной строчки в cyrtimes.sty
% Сточку не удалять!
% \usepackage{cyrtimes}

% Картнки и tikz
\usepackage{graphicx}
\usepackage{tikz}
\usetikzlibrary{snakes,arrows,shapes}


% Увы, поля придётся уменьшить из-за листингов.
\topmargin -1cm
\oddsidemargin -0.5cm
\evensidemargin -0.5cm
\textwidth 17cm
\textheight 24cm

\sloppy



% Оглавление в PDF
\usepackage[
bookmarks=true,
colorlinks=true, linkcolor=black, anchorcolor=black, citecolor=black, menucolor=black,filecolor=black, urlcolor=black,
unicode=true
]{hyperref}

% Для исходного кода в тексте
% \newcommand{\Code}[1]{\texttt{#1}}

% Некоторая русификация.
% \usepackage{misccorr} % Oh shi^W^W, оно не работает с report.
\usepackage{indentfirst}
\renewcommand{\labelitemi}{\normalfont\bfseries{--}}

% На дворе XXI век, но пакет listings всё ещё не пашет с русскими комментариями!

% Пакет listings для простой вставки исходников
% \usepackage{listings}
% Параметры оформления
% \lstset{
% showspaces=false,
% showtabs=false,
% frame=single,
% tabsize=4,
% basicstyle=\ttfamily,
% identifierstyle=\ttfamily,
% commentstyle=\itshape,
% stringstyle=\ttfamily,
% keywordstyle=\ttfamily,
% breaklines=true
% }
% Русский в комментариях.
% \lstset{escapebegin=\begin{cyr},escapeend=\end{cyr}}



% А это взято из файла, сгенерённого doxygen
%\usepackage{calc}
%\usepackage{array}
%\newenvironment{Code}
%{\footnotesize}
%{\normalsize}
%\newcommand{\doxyref}[3]{\textbf{#1} (\textnormal{#2}\,\pageref{#3})}
%\newenvironment{DocInclude}
%{\footnotesize}
%{\normalsize}
%\newenvironment{VerbInclude}
%{\footnotesize}
%{\normalsize}
%\newenvironment{Image}
%{\begin{figure}[H]}
%{\end{figure}}
%\newenvironment{ImageNoCaption}{}{}
%\newenvironment{CompactList}
%{\begin{list}{}{
%  \setlength{\leftmargin}{0.5cm}
%  \setlength{\itemsep}{0pt}
%  \setlength{\parsep}{0pt}
%  \setlength{\topsep}{0pt}
%  \renewcommand{\makelabel}{\hfill}}}
%{\end{list}}
%\newenvironment{CompactItemize}
%{
%  \begin{itemize}
%  \setlength{\itemsep}{-3pt}
%  \setlength{\parsep}{0pt}
%  \setlength{\topsep}{0pt}
%  \setlength{\partopsep}{0pt}
%}
%{\end{itemize}}
%\newcommand{\PBS}[1]{\let\temp=\\#1\let\\=\temp}
%\newlength{\tmplength}
%\newenvironment{TabularC}[1]
%{
%\setlength{\tmplength}
%     {\linewidth/(#1)-\tabcolsep*2-\arrayrulewidth*(#1+1)/(#1)}
%      \par\begin{tabular*}{\linewidth}
%             {*{#1}{|>{\PBS\raggedright\hspace{0pt}}p{\the\tmplength}}|}
%}
%{\end{tabular*}\par}
%\newcommand{\entrylabel}[1]{
%   {\parbox[b]{\labelwidth-4pt}{\makebox[0pt][l]{\textbf{#1}}\vspace{1.5\baselineskip}}}}
%\newenvironment{Desc}
%{\begin{list}{}
%  {
%    \settowidth{\labelwidth}{40pt}
%    \setlength{\leftmargin}{\labelwidth}
%    \setlength{\parsep}{0pt}
%    \setlength{\itemsep}{-4pt}
%    \renewcommand{\makelabel}{\entrylabel}
%  }
%}
%{\end{list}}
%\newenvironment{Indent}
%  {\begin{list}{}{\setlength{\leftmargin}{0.5cm}}
%      \item[]\ignorespaces}
%  {\unskip\end{list}}
%


\title{Отчёт \\ по лабораторной работе №2 \\ на тему: \\ "Моделирование мьютексов" \\ по курсу: Математические основы верификации ПО}
\author{Жаровой Наталии Александровны, ИУ7-41М\\}

\begin{document}

\maketitle

\section*{Постановка задачи}

На языке Promela описать взаимодействие двух процессов, разделяющих ресурсы. После обнаружения гонки необходимо 
смоделировать использование мьютекса для предотвращения гонки.

\section*{Моделирование состояния гонки}
Для того чтобы смоделировать состояние гонки, создадим глобальный объект счётчика <<x>> и два
конкурирующих процесса. При этом процесс <<counter>> будет увеличивать счётчик на 1, а процесс <<printer>>
будет выводить текущее значение счётчика, в том случае если его значение нечётно.

\subsection*{Листинг программы}
\begin{Verbatim}
#define N 10

int x = 0
bool end = 0

proctype counter() {
    do
    :: (x > N) -> break
    :: x = x + 1
    od;
    end = 1
}

proctype printer() {
    do
    :: (end) -> break
    ::  if
            :: (x % 2) -> printf("x=%d\n", x)
            :: skip
        fi
    od
}

init {
    run printer();
    run counter();
}
\end{Verbatim}

Если несколько раз позапускать данную модель, можно увидеть, что не всегда выводятся только нечётные значения счётчика:
\begin{Verbatim}
          x=2
          x=11
          x=11
3 processes created
\end{Verbatim}

Используя возможность языка Promela <<atomic>>, определим проверку на нечётность и непосредственно вывод значения как атомарную последовательность.
Таким образом, второй поток не сможет поменять значение глобальной переменной между проверкой и выводом.

\subsection*{Листинг исправленной программы}
\begin{Verbatim}
#define N 10

int x = 0
bool end = 0

proctype counter() {
    do
    :: (x > N) -> break
    :: x = x + 1
    od;
    end = 1
}

proctype printer() {
    do
    :: (end) -> break
    ::  if
            :: atomic { (x % 2) -> printf("x=%d\n", x) }
            :: skip
        fi
    od
}

init {
    run printer();
    run counter();
}
\end{Verbatim}

Теперь сколько бы раз мы не запускали модель, всегда выводятся только нечётные значения счётчика.

\section*{Выводы}

В процессе выполнения работы была написана модель программы, допускающей состояние гонки. С использованием возможности языка Promela <<atomic>> 
гонка была устранена, путём объединения нескольких операций в неделимую последовательность.

\end{document}
