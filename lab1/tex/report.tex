\documentclass[a4paper,12pt]{report}

\usepackage[utf8x]{inputenc}
\usepackage[T2A]{fontenc}
\usepackage[english, russian]{babel}
\usepackage{doxygen}


% Опционно, требует  apt-get install scalable-cyrfonts.*
% и удаления одной строчки в cyrtimes.sty
% Сточку не удалять!
% \usepackage{cyrtimes}

% Картнки и tikz
\usepackage{graphicx}
\usepackage{tikz}
\usetikzlibrary{snakes,arrows,shapes}


% Увы, поля придётся уменьшить из-за листингов.
\topmargin -1cm
\oddsidemargin -0.5cm
\evensidemargin -0.5cm
\textwidth 17cm
\textheight 24cm

\sloppy



% Оглавление в PDF
\usepackage[
bookmarks=true,
colorlinks=true, linkcolor=black, anchorcolor=black, citecolor=black, menucolor=black,filecolor=black, urlcolor=black,
unicode=true
]{hyperref}

% Для исходного кода в тексте
% \newcommand{\Code}[1]{\texttt{#1}}

% Некоторая русификация.
% \usepackage{misccorr} % Oh shi^W^W, оно не работает с report.
\usepackage{indentfirst}
\renewcommand{\labelitemi}{\normalfont\bfseries{--}}

% На дворе XXI век, но пакет listings всё ещё не пашет с русскими комментариями!

% Пакет listings для простой вставки исходников
% \usepackage{listings}
% Параметры оформления
% \lstset{
% showspaces=false,
% showtabs=false,
% frame=single,
% tabsize=4,
% basicstyle=\ttfamily,
% identifierstyle=\ttfamily,
% commentstyle=\itshape,
% stringstyle=\ttfamily,
% keywordstyle=\ttfamily,
% breaklines=true
% }
% Русский в комментариях.
% \lstset{escapebegin=\begin{cyr},escapeend=\end{cyr}}



% А это взято из файла, сгенерённого doxygen
%\usepackage{calc}
%\usepackage{array}
%\newenvironment{Code}
%{\footnotesize}
%{\normalsize}
%\newcommand{\doxyref}[3]{\textbf{#1} (\textnormal{#2}\,\pageref{#3})}
%\newenvironment{DocInclude}
%{\footnotesize}
%{\normalsize}
%\newenvironment{VerbInclude}
%{\footnotesize}
%{\normalsize}
%\newenvironment{Image}
%{\begin{figure}[H]}
%{\end{figure}}
%\newenvironment{ImageNoCaption}{}{}
%\newenvironment{CompactList}
%{\begin{list}{}{
%  \setlength{\leftmargin}{0.5cm}
%  \setlength{\itemsep}{0pt}
%  \setlength{\parsep}{0pt}
%  \setlength{\topsep}{0pt}
%  \renewcommand{\makelabel}{\hfill}}}
%{\end{list}}
%\newenvironment{CompactItemize}
%{
%  \begin{itemize}
%  \setlength{\itemsep}{-3pt}
%  \setlength{\parsep}{0pt}
%  \setlength{\topsep}{0pt}
%  \setlength{\partopsep}{0pt}
%}
%{\end{itemize}}
%\newcommand{\PBS}[1]{\let\temp=\\#1\let\\=\temp}
%\newlength{\tmplength}
%\newenvironment{TabularC}[1]
%{
%\setlength{\tmplength}
%     {\linewidth/(#1)-\tabcolsep*2-\arrayrulewidth*(#1+1)/(#1)}
%      \par\begin{tabular*}{\linewidth}
%             {*{#1}{|>{\PBS\raggedright\hspace{0pt}}p{\the\tmplength}}|}
%}
%{\end{tabular*}\par}
%\newcommand{\entrylabel}[1]{
%   {\parbox[b]{\labelwidth-4pt}{\makebox[0pt][l]{\textbf{#1}}\vspace{1.5\baselineskip}}}}
%\newenvironment{Desc}
%{\begin{list}{}
%  {
%    \settowidth{\labelwidth}{40pt}
%    \setlength{\leftmargin}{\labelwidth}
%    \setlength{\parsep}{0pt}
%    \setlength{\itemsep}{-4pt}
%    \renewcommand{\makelabel}{\entrylabel}
%  }
%}
%{\end{list}}
%\newenvironment{Indent}
%  {\begin{list}{}{\setlength{\leftmargin}{0.5cm}}
%      \item[]\ignorespaces}
%  {\unskip\end{list}}
%


\title{Отчёт \\ по лабораторной работе №1 \\ на тему: \\ "Знакомство с Promela" \\ по курсу: Математические основы верификации ПО}
\author{Жаровой Наталии Александровны, ИУ7-41М\\}

\begin{document}

\maketitle

\section*{Постановка задачи}

Необходимо на языке Promela описать модель программы, осуществляющей переход 
в разные состояния. Отчет должен содержать код на языке promela и схему 
автомата переходов между состояниями.

\section*{Листинг программы}
\begin{Verbatim}
#define N 6

proctype fibonacci(chan p; byte n){
    int f1 = 0;
    int f2 = 1;
    int result = 0;
    byte current_n = 1;
    do
    :: (n == 0) -> break
    :: (n == 1) ->
        result = f2;
        break 
    :: (current_n < n) ->
        result = f1 + f2;
        f1 = f2;
        f2 = result;
        current_n = current_n + 1
    :: (current_n == n) -> break
    od;
    p!result
}


init {
    chan child = [1] of {int}
    int result;

    run fibonacci(child, N);
    child?result;
    printf("result: %d\n", result)
}

\end{Verbatim}

\section*{Вывод программы}
\begin{Verbatim}
spin -p -l  lab1.pml

  0:    proc  - (:root:) creates proc  0 (:init:)
Starting fibonacci with pid 1
  1:    proc  0 (:init::1) creates proc  1 (fibonacci)
  1:    proc  0 (:init::1) lab1.pml:28 (state 1)        [(run fibonacci(child,6))]
                queue 1 (:init:(0):child): 
  2:    proc  1 (fibonacci:1) lab1.pml:13 (state 6)     [((current_n<n))]
                queue 1 (fibonacci(1):p): 
  3:    proc  1 (fibonacci:1) lab1.pml:14 (state 7)     [result = (f1+f2)]
                fibonacci(1):result = 1
                queue 1 (fibonacci(1):p): 
  4:    proc  1 (fibonacci:1) lab1.pml:15 (state 8)     [f1 = f2]
                fibonacci(1):f1 = 1
                queue 1 (fibonacci(1):p): 
  5:    proc  1 (fibonacci:1) lab1.pml:16 (state 9)     [f2 = result]
                fibonacci(1):f2 = 1
                queue 1 (fibonacci(1):p): 
  6:    proc  1 (fibonacci:1) lab1.pml:17 (state 10)    [current_n = (current_n+1)]
                fibonacci(1):current_n = 2
                queue 1 (fibonacci(1):p): 
  7:    proc  1 (fibonacci:1) lab1.pml:20 (state 14)    [.(goto)]
                queue 1 (fibonacci(1):p): 
  8:    proc  1 (fibonacci:1) lab1.pml:13 (state 6)     [((current_n<n))]
                queue 1 (fibonacci(1):p): 
  9:    proc  1 (fibonacci:1) lab1.pml:14 (state 7)     [result = (f1+f2)]
                fibonacci(1):result = 2
                queue 1 (fibonacci(1):p): 
 10:    proc  1 (fibonacci:1) lab1.pml:15 (state 8)     [f1 = f2]
                fibonacci(1):f1 = 1
                queue 1 (fibonacci(1):p): 
 11:    proc  1 (fibonacci:1) lab1.pml:16 (state 9)     [f2 = result]
                fibonacci(1):f2 = 2
                queue 1 (fibonacci(1):p): 
 12:    proc  1 (fibonacci:1) lab1.pml:17 (state 10)    [current_n = (current_n+1)]
                fibonacci(1):current_n = 3
                queue 1 (fibonacci(1):p): 
 13:    proc  1 (fibonacci:1) lab1.pml:20 (state 14)    [.(goto)]
                queue 1 (fibonacci(1):p): 
 14:    proc  1 (fibonacci:1) lab1.pml:13 (state 6)     [((current_n<n))]
                queue 1 (fibonacci(1):p): 
 15:    proc  1 (fibonacci:1) lab1.pml:14 (state 7)     [result = (f1+f2)]
                fibonacci(1):result = 3
                queue 1 (fibonacci(1):p): 
 16:    proc  1 (fibonacci:1) lab1.pml:15 (state 8)     [f1 = f2]
                fibonacci(1):f1 = 2
                queue 1 (fibonacci(1):p): 
 17:    proc  1 (fibonacci:1) lab1.pml:16 (state 9)     [f2 = result]
                fibonacci(1):f2 = 3
                queue 1 (fibonacci(1):p): 
 18:    proc  1 (fibonacci:1) lab1.pml:17 (state 10)    [current_n = (current_n+1)]
                fibonacci(1):current_n = 4
                queue 1 (fibonacci(1):p): 
 19:    proc  1 (fibonacci:1) lab1.pml:20 (state 14)    [.(goto)]
                queue 1 (fibonacci(1):p): 
 20:    proc  1 (fibonacci:1) lab1.pml:13 (state 6)     [((current_n<n))]
                queue 1 (fibonacci(1):p): 
 21:    proc  1 (fibonacci:1) lab1.pml:14 (state 7)     [result = (f1+f2)]
                fibonacci(1):result = 5
                queue 1 (fibonacci(1):p): 
 22:    proc  1 (fibonacci:1) lab1.pml:15 (state 8)     [f1 = f2]
                fibonacci(1):f1 = 3
                queue 1 (fibonacci(1):p): 
 23:    proc  1 (fibonacci:1) lab1.pml:16 (state 9)     [f2 = result]
                fibonacci(1):f2 = 5
                queue 1 (fibonacci(1):p): 
 24:    proc  1 (fibonacci:1) lab1.pml:17 (state 10)    [current_n = (current_n+1)]
                fibonacci(1):current_n = 5
                queue 1 (fibonacci(1):p): 
 25:    proc  1 (fibonacci:1) lab1.pml:20 (state 14)    [.(goto)]
                queue 1 (fibonacci(1):p): 
 26:    proc  1 (fibonacci:1) lab1.pml:13 (state 6)     [((current_n<n))]
                queue 1 (fibonacci(1):p): 
 27:    proc  1 (fibonacci:1) lab1.pml:14 (state 7)     [result = (f1+f2)]
                fibonacci(1):result = 8
                queue 1 (fibonacci(1):p): 
 28:    proc  1 (fibonacci:1) lab1.pml:15 (state 8)     [f1 = f2]
                fibonacci(1):f1 = 5
                queue 1 (fibonacci(1):p): 
 29:    proc  1 (fibonacci:1) lab1.pml:16 (state 9)     [f2 = result]
                fibonacci(1):f2 = 8
                queue 1 (fibonacci(1):p): 
 30:    proc  1 (fibonacci:1) lab1.pml:17 (state 10)    [current_n = (current_n+1)]
                fibonacci(1):current_n = 6
                queue 1 (fibonacci(1):p): 
 31:    proc  1 (fibonacci:1) lab1.pml:20 (state 14)    [.(goto)]
                queue 1 (fibonacci(1):p): 
 32:    proc  1 (fibonacci:1) lab1.pml:18 (state 11)    [((current_n==n))]
                queue 1 (fibonacci(1):p): 
 33:    proc  1 (fibonacci:1) lab1.pml:18 (state 12)    [goto :b0]
                queue 1 (fibonacci(1):p): 
 34:    proc  1 (fibonacci:1) lab1.pml:20 (state 16)    [p!result]
                queue 1 (fibonacci(1):p): [8]
 34:    proc  1 (fibonacci:1)           terminates
 35:    proc  0 (:init::1) lab1.pml:29 (state 2)        [child?result]
                :init:(0):result = 8
                queue 1 (:init:(0):child): 
      result: 8
 36:    proc  0 (:init::1) lab1.pml:30 (state 3)    [printf('result: %d\\n',result)]
                queue 1 (:init:(0):child): 
 36:    proc  0 (:init::1)       terminates
2 processes created
\end{Verbatim}

\section*{Схемы переходов между состояниями программы}
На рисунках \ref{fig:init}, \ref{fig:p_fibonacci} изображены схемы переходов между сотстояниями для каждого процесса: 

\begin{figure}
\centering
\includegraphics[]{../bin/init_dot.png}
\caption{Процесс init}
\label{fig:init}
\end{figure}

\begin{figure}
\centering
\includegraphics[]{../bin/p_fibonacci_dot.png}
\caption{Процесс fibonacci}
\label{fig:p_fibonacci}
\end{figure}

\section*{Выводы}

В процессе выполнения работы была написана модель программы, 
вычисляющей N-ый элемент последовательности Фибоначчи, на языке Promela. 
Так же были изучены основные принципы работы с программой spin для анализа 
моделей на языке Promela и сгенерирована схема переходов между состояниями программы.


\end{document}
